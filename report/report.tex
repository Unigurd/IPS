\documentclass[oneside, a4paper, 11pt]{article} % onesided a4paper med 11pt font.
\usepackage[english,danish]{babel} 
\usepackage[utf8]{inputenc}
\usepackage[T1]{fontenc}
\usepackage[a4paper,top=3cm,bottom=2cm,left=3cm,right=3cm,marginparwidth=1.75cm]{geometry} % sets margin.
\usepackage[table,xcdraw]{xcolor}
\usepackage[normalem]{ulem}
\useunder{\uline}{\ul}{}
\usepackage{tabu}  %%Til at lave skemaer
\usepackage{mathtools}
\usepackage{blindtext}
\usepackage{enumitem}
\usepackage{amsmath, calc}
\usepackage{graphicx}
\usepackage[colorinlistoftodos]{todonotes}
\usepackage[colorlinks=true, allcolors=blue]{hyperref}
\usepackage{amsfonts,amssymb,amsthm}
\usepackage{float}
\usepackage{bm} % bold package
\usepackage{multirow}
\usepackage{xfrac}
\usepackage{moreverb}
\usepackage{booktabs}
\usepackage{tikz}
%\usepackage{soul}
\usepackage{centernot}
\def\checkmark{\tikz\fill[scale=0.4](0,.35) -- (.25,0) -- (1,.7) -- (.25,.15) -- cycle;} 
\usepackage{siunitx} % aktiverer genveje til forskellige talsymboler, eksempelvis bliver \R == de reele tal.
	\newcommand{\N}{\ensuremath{\mathbb{N}}}
	\newcommand{\Z}{\ensuremath{\mathbb{Z}}}
	\newcommand{\Q}{\ensuremath{\mathbb{Q}}}
	\newcommand{\R}{\ensuremath{\mathbb{R}}}
	\newcommand{\C}{\ensuremath{\mathbb{C}}}
	\newcommand{\F}{\ensuremath{\mathbb{F}}}
\usepackage[affil-it]{authblk} %% AFFILIATION PACKAGE. %%
\DeclarePairedDelimiter\ceil{\lceil}{\rceil} % aktiverer genveje til mathCeil og mathFloor.
\DeclarePairedDelimiter\floor{\lfloor}{\rfloor}
\usepackage{listings}
\lstset{
  basicstyle=\ttfamily,
  breaklines=true,
  mathescape,
  numbers=left,
  stepnumber=1,
  columns=fullflexible,
  keepspaces=true,
  inputencoding=ansinew,
  extendedchars=\true,
  escapeinside=__,
  literate=%
{æ}{{\ae}}1
{å}{{\aa}}1
{ø}{{\o}}1
{Æ}{{\AE}}1
{Å}{{\AA}}1
{Ø}{{\O}}1,
}
\let\origthelstnumber\thelstnumber
\makeatletter
\newcommand*\Suppressnumber{%
  \lst@AddToHook{OnNewLine}{%
    \let\thelstnumber\relax%
     \advance\c@lstnumber-\@ne\relax%
    }
}
\newcommand*\Reactivatenumber{%
  \lst@AddToHook{OnNewLine}{%
   \let\thelstnumber\origthelstnumber%
   \advance\c@lstnumber\@ne\relax}%
}

\title{IPS Group Project}
\author{Sigurd Sonniks (lsn777) \and Jesper Rabjerg (lxc355)}
\date{\today}

\begin{document}

\maketitle
\pagebreak
\section*{Task 1}
\subsection*{Part a}

For this part, we were tasked with implementing multiplication, division, integer negation, boolean negation, $\&\&$, $||$, true and false. \\

The following simple regular expressions were added to the lexer to facilitate this. All of the matches attach the position in the file to the constructed token, and "true" and "false" attach their value as an F\# value as well.
\begin{verbatim}
  | '~'                   { Parser.NEGATE (getPos lexbuf) }
  | "&&"                  { Parser.AND    (getPos lexbuf) }
  | "||"                  { Parser.OR     (getPos lexbuf) }
  | '*'                   { Parser.TIMES  (getPos lexbuf) }
  | '/'                   { Parser.DIVIDE (getPos lexbuf) }
  | "true"                { Parser.TRUE   (true,  getPos lexbuf) }
  | "false"               { Parser.FALSE  (false, getPos lexbuf) }
  | "not"                 { Parser.NOT    (getPos lexbuf) }
\end{verbatim}
~\\

In the parser we added the corresponding tokens:
\begin{verbatim}
%token <bool*(int*int)> TRUE FALSE
%token <(int*int)> AND OR NOT NEGATE
%token <(int*int)> TIMES DIVIDE
\end{verbatim}
~\\

We also added the following precedence rules in order to negate integers before multiplication and division, which in turn should be before addition and multiplication. But since $-(a \dot b) = (-a) \dot b = a \dot (-b)$, and likewise with addition, it is not important that negation happens before multiplication/division, just that it happens before addition and subtraction. Note also the binary operators all are left-associative as specified. \\

For the boolean operators, negation happens before and'ing, which happens before or'ing.
\begin{verbatim}
%nonassoc ifprec letprec
%left DEQ LTH
%left PLUS MINUS
%left TIMES DIVIDE
%nonassoc NEGATE

%left OR
%left AND
%nonassoc NOT
\end{verbatim}

The following cases were added to the Exp rule to add the new operations to the syntax tree. All of the F\# types take a position in the source file as their last argument. Furthermore, when constructing constant booleans they take their value as well. Unary operators take an expression that is the expression they operate on, and binary operators take two expressions, which, unsurprisingly, are the two expression they operate on.
\begin{verbatim}
    | TRUE           { Constant (BoolVal (fst $1), snd $1) }
    | FALSE          { Constant (BoolVal (fst $1), snd $1) }
    | Exp TIMES Exp  { Times($1, $3, $2) }
    | Exp DIVIDE Exp { Divide($1, $3, $2) }
    | Exp AND Exp    { And ($1, $3, $2) }
    | Exp OR Exp     { Or ($1, $3, $2) }
    | NOT Exp        { Not ($2, $1) }
    | NEGATE Exp     { Negate($2, $1) }
 \end{verbatim} 


In the interpreter we added cases to evalExp for each of the operations. \\

For multiplication, we evaluate each of the subexpressions. If both of them are integers, we multiply them and return the result. If any of them are anything else, we error. The same approach was used for division so the code will not be shown here.
\begin{verbatim}
  | Times(e1, e2, pos) ->
      let res1 = evalExp(e1,vtab, ftab)
      let res2 = evalExp(e2,vtab, ftab)
      match (res1, res2) with
          | (IntVal n1, IntVal n2) -> IntVal (n1*n2)
          | _ -> invalidOperands "Multiplication on non-integral args: " [(Int, Int)] res1 res2 pos
\end{verbatim}

\begin{comment}
  | Divide(e1, e2, pos) ->
      let res1 = evalExp(e1,vtab, ftab)
      let res2 = evalExp(e2,vtab, ftab)
      match (res1, res2) with
          | (IntVal n1, IntVal n2) -> IntVal (n1/n2)
          | _ -> invalidOperands "Division on non-integral args: " [(Int, Int)] res1 res2 pos
\end{comment}
~\\

For and'ing, we evaluate the two expressions in steps. This is because if the first one turns out to be false, the and-expression cannot possibly be true and we thus don't need to evaluate the second expression, the result of which is thus set to false. The result of the second expression can thus only be true if both this and the first expression evaluated to true, and we can thus safely return that. But if either of the first or second evaluated to something else than a boolean, we set typeError1 or typeError2, respectively, to true. If any of them are true, we error with the appropriate expression. The same approach was used for or'ing, so the code is not shown here.
\begin{verbatim}
  | And (e1, e2, pos) ->
      let res1 = evalExp(e1,vtab,ftab)
      let (typeError1,res2) =
          match res1 with
              | BoolVal true  -> (false, evalExp(e2,vtab,ftab))
              | BoolVal false -> (false, (BoolVal false))
              | _             -> (true, (BoolVal false))
      let typeError2 =
          match res2 with
              |BoolVal _ -> false
              |_         -> true

      if typeError1 then
          invalidOperand "and'ing of non-bool arg: " Bool res1 pos
      elif typeError2 then
          invalidOperand "and'ing of non-bool arg: " Bool res2 pos
      else
          res2
\end{verbatim}
~\\
\begin{comment}
  | Or (e1, e2, pos) ->
      let res1 = evalExp(e1,vtab,ftab)
      let (typeError1,res2) =
          match res1 with
              | BoolVal true  -> (false, (BoolVal true))
              | BoolVal false -> (false, evalExp(e2,vtab,ftab))
              | _             -> (true, (BoolVal false))
      let typeError2 =
          match res2 with
              |BoolVal _ -> false
              |_         -> true

      if typeError1 then
          invalidOperand "and'ing of non-bool arg: " Bool res1 pos
      elif typeError2 then
          invalidOperand "and'ing of non-bool arg: " Bool res2 pos
      else
          res2
\end{comment}

For negation, we evaluate the expression and match it with the proper type. If it is of the proper type we negate it, and otherwise we report the error. The same approach was used for both boolean and integer negation, and so only the boolean variant is shown here. 
\begin{verbatim}
  | Not(e1, pos) ->
      match res1 with
      let res1 = evalExp(e1,vtab, ftab)
          |BoolVal n1 -> BoolVal (not n1)
          |_ -> invalidOperand "Negation of non-boolean: " Bool res1 pos
\end{verbatim}
~\\
\begin{comment}
  //cannot test negation with multiplication and division
  | Negate(e1, pos) ->
      let res1 = evalExp(e1,vtab, ftab)
      match res1 with
          |IntVal n1 -> IntVal (-n1)
          |_ -> invalidOperand "Negation of non-boolean: " Bool res1 pos
\end{comment}

For type checking and, or, multiplication and division, we check the types of the argument expressions. If they have the proper types we return the result type along with the expression. If they don't we raise an appropriate error. Multiplication is shown here. The other operations are implemented in a similar fashion.
\begin{verbatim}
    | Times (e1, e2, pos) ->
        let (t1, e1_dec) = checkExp ftab vtab e1
        let (t2, e2_dec) = checkExp ftab vtab e2
        if (Int = t1 && Int = t2)
        then (Int, Times (e1_dec, e2_dec, pos))
        else raise (MyError ("In Times: one of subexpression types is not Int: "+ppType t1+" and "+ppType t2, pos))
\end{verbatim}
~\\
\begin{comment}
    | Divide (e1, e2, pos) ->
        let (t1, e1_dec) = checkExp ftab vtab e1
        let (t2, e2_dec) = checkExp ftab vtab e2
        if (Int = t1 && Int = t2)
        then (Int, Divide(e1_dec, e2_dec, pos))
        else raise (MyError ("In Divide: one of subexpression types is not Int: "+ppType t1+" and "+ppType t2, pos))

    | And (e1, e2, pos) ->
        let (t1, e1_dec) = checkExp ftab vtab e1
        let (t2, e2_dec) = checkExp ftab vtab e2
        if (Bool = t1 && Bool = t2)
        then (Bool, And(e1_dec, e2_dec, pos))
        else raise (MyError ("In And: one of subexpression types is not Bool: "+ppType t1+" and "+ppType t2, pos))


    | Or (e1, e2, pos) ->
        let (t1, e1_dec) = checkExp ftab vtab e1
        let (t2, e2_dec) = checkExp ftab vtab e2
        if (Bool = t1 && Bool = t2)
        then (Bool, Or(e1_dec, e2_dec, pos))
        else raise (MyError ("In Or: one of subexpression types is not Bool: "+ppType t1+" and "+ppType t2, pos))
\end{comment}

Integer and boolean negation are implemented in a similar fashion. Since they only take one expression as an argument, we just type check that expression. Boolean negation is shown here, but integer negation is implemented in the same way.
\begin{verbatim}
    | Not (e1, pos) ->
        let (t1, e1_dec) = checkExp ftab vtab e1
        if (Bool = t1) 
        then (Bool, Not(e1_dec, pos))
        else raise (MyError ("In Not: subexpression type is not Bool: "+ppType t1, pos))
\end{verbatim}

\begin{comment}
    | Negate (e1, pos) ->
        let (t1, e1_dec) = checkExp ftab vtab e1
        if (Int= t1) 
        then (Int, Negate(e1_dec, pos))
        else raise (MyError ("In Negate: subexpression type is not Int: "+ppType t1, pos))
\end{comment}

\subsection*{Part b}

For implementing multiple-declaration let statements, we added a regular expression that matches a semi-colon and constructs a SEMICOLON token in the lexer.
\begin{verbatim}
  | ';'                   { Parser.SEMICOLON   (getPos lexbuf) }
\end{verbatim}
~\\

We added a new non-terminal to facilitate multi-lets. This non-terminal can essentially be thought of as a linked list of lets separated by semicolons. It either matches a declaration with a semicolon and some more declarations afterwards or a final declaration with an expression to use it in afterwards. This ensures that there will always be at least one declaration in a let statement.
\begin{verbatim}
Decs : ID EQ Exp SEMICOLON Decs
                    { Let (Dec (fst $1, $3, $2), $5, snd $1) }
    | ID EQ Exp IN Exp
                    { Let (Dec (fst $1, $3, $2), $5, snd $1) }
\end{verbatim}
~\\

The let rule in the Exp non-terminal is changed to use the new Decs non-terminal.
\begin{verbatim}
    | LET Decs %prec letprec
\end{verbatim}




\pagebreak
\section*{Task 2}

\subsection*{replicate}

We added the string "replicate" to the keyword function in order to create a REPLICATE token.
\begin{verbatim}
       | "replicate"    -> Parser.REPLICATE pos
\end{verbatim}
~\\

We then added a rule to the Exp non-terminal. It matches a replicate token given proper arguments and builds a syntax tree from the information.
\begin{verbatim}
    | REPLICATE LPAR Exp COMMA Exp RPAR
                     { Replicate ($3, $5, (), $1) } 
\end{verbatim}
~\\

In the interpreter we evaluate the expressions yielding the size and the element to replicate. If sz is an int we create a list of correct size and map over it to change all elements to elm. Otherwise we raise an error.
\begin{verbatim}
  | Replicate (n, a, _, pos) ->
       let sz  = evalExp(n, vtab, ftab)
       let elm = evalExp(a, vtab, ftab)
       match sz with
          | IntVal size ->
              if size >= 0
              then ArrayVal( List.map (fun x -> elm) [0..size-1], Int )
              else let msg = sprintf "Error: In replicate call, size is negative: %i" size
                   raise (MyError(msg, pos))
          | _ -> raise (MyError("replicate argument is not a number: "+ppVal 0 sz, pos))
\end{verbatim}
~\\

When typechecking, we check the size-expression, and if it is of type Int we return a tuple consisting ofArray (elm\_type), where elm\_type is the type of the elements in the array, and the typed syntax tree.
\begin{verbatim}
    | Replicate (n, a, _, pos) ->
        let (sz_type, sz_exp_dec) = checkExp ftab vtab n
        if sz_type = Int then
            let (elm_type, elm_exp_dec) = checkExp ftab vtab a
            (Array (elm_type), 
                    Replicate (sz_exp_dec, elm_exp_dec, elm_type, pos))
        else 
           raise (MyError ("replicate: Argument not an int", pos))
\end{verbatim}
~\\

The translation of replicate to MIPS was based on the translation of iota, so we will not show all the code here. The essence of the translated MIPS is that the expression that calculates the size of the array is evaluated. If it is too small an error is raised, and otherwise the array is allocated. Then the expression that evaluates the thing to be copied into the array is evaluated. Next is a loop that copies that thing into each index of the array. \\

We will now show and explain the changes from the \texttt{iota} code. \\
The first change is that the expression that results in the value to be copied needs to be translated. The result is saved in register val\_reg and the translation is in a\_code.
\begin{verbatim}
      let val_reg  = newName "val_reg"
      let a_code = compileExp a_exp vtable val_reg
\end{verbatim}
~\\

Booleans and characters only take up 1 byte instead of the usual 4, so we need to deal with these cases. We pattern match on the size of the elements in the array. If the size is 1 we only save 1 byte at a time.
\begin{verbatim}
     let loop_replicate =
          match getElemSize elem_type with
              | One  -> [ Mips.SB (val_reg, addr_reg, "0") ]
              | Four -> [ Mips.SW (val_reg, addr_reg, "0") ]
\end{verbatim}
~\\

Likewise, if the size is 1 we only add 1 to the address where we copy to, while we add 4 to it if the size is 4, i.e. a word. 
\begin{verbatim}
      let loop_footer = 
          match getElemSize elem_type with
              | One  ->
                  [ Mips.ADDI (addr_reg, addr_reg, "1")
                  ; Mips.ADDI (i_reg, i_reg, "1")
                  ; Mips.J loop_beg
                  ; Mips.LABEL loop_end
                  ]
 
              | Four ->
                  [ Mips.ADDI (addr_reg, addr_reg, "4")
                  ; Mips.ADDI (i_reg, i_reg, "1")
                  ; Mips.J loop_beg
                  ; Mips.LABEL loop_end
                  ]
\end{verbatim}
~\\

Lastly we concatenate all these pieces of MIPS code together like we described earlier.
\begin{verbatim}
      n_code
       @ checksize
       @ dynalloc (size_reg, place, elem_type)
       @ a_code
       @ init_regs
       @ loop_header
       @ loop_replicate
       @ loop_footer
\end{verbatim}


\pagebreak
\section*{Task 3}

\begin{verbatim}
        | Index (name, e, t, pos) ->
            (* TODO project task 3:
                Should probably do the same as the `Var` case, for
                the array name, and optimize the index expression `e` as well.
            *)

            let e' = copyConstPropFoldExp vtable e
            match SymTab.lookup name vtable with
                | Some a ->
                    match a with
                    | ConstProp c -> 
                        match (c,e') with
                            | ArrayVal (exp_lst, _), Constant (IntVal i, pos) -> 
                                Constant (exp_lst.[i], pos)
                            | _, _ -> Index (name, e', t, pos)

                    | VarProp v   -> Index (v, e', t, pos)

                | None   -> Index (name, e', t, pos)



        | Let (Dec (name, e, decpos), body, pos) ->
            let e' = copyConstPropFoldExp vtable e
            match e' with
                | Var (v, _) ->
                    (* TODO project task 3:
                        Hint: I have discovered a variable-copy statement `let x = a`.
                              I should probably record it in the `vtable` by
                              associating `x` with a variable-propagatee binding,
                              and optimize the `body` of the let.
                    *)
                    copyConstPropFoldExp (SymTab.bind name (VarProp v) vtable) body

                | Constant (c, pos) ->
                    copyConstPropFoldExp (SymTab.bind name (ConstProp c) vtable) body

                | Let (dec2, body2, pos2) ->
                    (* TODO project task 3:
                        Hint: this has the structure
                                `let y = (let x = e1 in e2) in e3`
                        Problem is, in this form, `e2` may simplify
                        to a variable or constant, but I will miss
                        identifying the resulting variable/constant-copy
                        statement on `y`.
                        A potential solution is to optimize directly the
                        restructured, semantically-equivalent expression:
                                `let x = e1 in let y = e2 in e3`
                    *)
                    copyConstPropFoldExp vtable (Let (dec2, Let (Dec (name, body2, decpos), body, pos), pos2))
\end{verbatim}

\begin{verbatim}
        | Times (e1, e2, pos) ->
            (* TODO project task 3: implement as many safe algebraic
                simplifications as you can think of. You may inspire 
                yourself from the case of `Plus`. For example:
                     1 * x = ? 
                     x * 0 = ?
            *)
            let e1' = copyConstPropFoldExp vtable e1
            let e2' = copyConstPropFoldExp vtable e2
            match (e1', e2') with
                | (Constant (IntVal x, _), Constant (IntVal y, _)) ->
                    Constant (IntVal (x * y), pos)
                | (Constant (IntVal 0, _), _) -> Constant (IntVal 0, pos)
                | (_, Constant (IntVal 0, _)) -> Constant (IntVal 0, pos)
                | (Constant (IntVal 1, _), _) -> e2'
                | (_, Constant (IntVal 1, _)) -> e1'
                | _ -> Times (e1', e2', pos)
 
        | And (e1, e2, pos) ->
            (* TODO project task 3: see above. you may inspire yourself from `Or` *)

            let e1' = copyConstPropFoldExp vtable e1
            let e2' = copyConstPropFoldExp vtable e2
            match (e1', e2') with
                | Constant (BoolVal a, _), Constant (BoolVal b, _) ->
                    Constant (BoolVal (a && b), pos)
                | Constant (BoolVal true, _), b  -> b
                | a, Constant (BoolVal true, _)  -> a
                | Constant (BoolVal false, _), _ -> Constant (BoolVal false, pos)
                | _, Constant (BoolVal false, _) -> Constant (BoolVal false, pos)
                | _ -> And (e1', e2', pos)
\end{verbatim}



\begin{verbatim}
        | Var (name, pos) ->
           (false, SymTab.fromList [name,()], Var (name, pos))

        | Index (name, e, t, pos) ->
           let (ios, uses, e') = removeDeadBindingsInExp e
            (ios, SymTab.bind name () uses, Index (name, e', t, pos)) 

        | Let (Dec (name, e, decpos), body, pos) ->
           let (bodyios, bodyuses, body') = removeDeadBindingsInExp body
            match SymTab.lookup name bodyuses with
                | Some _ ->
                    let eios, euses, e' = removeDeadBindingsInExp e
                    (eios || bodyios, SymTab.combine bodyuses euses, Let (Dec (name, e', decpos), body', pos))
                | None -> (bodyios, bodyuses, body') 
\end{verbatim}

\pagebreak
\section*{Task 4}

\end{document}

