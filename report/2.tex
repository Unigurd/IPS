\pagebreak
\section*{Task 2}

\subsection*{replicate}

We added the string "replicate" to the keyword function in order to create a REPLICATE token.
\begin{verbatim}
       | "replicate"    -> Parser.REPLICATE pos
\end{verbatim}
~\\

We then added a rule to the Exp non-terminal. It matches a replicate token given proper arguments and builds a syntax tree from the information.
\begin{verbatim}
    | REPLICATE LPAR Exp COMMA Exp RPAR
                     { Replicate ($3, $5, (), $1) } 
\end{verbatim}
~\\

In the interpreter we evaluate the expressions yielding the size and the element to replicate. If sz is an int we create a list of correct size and map over it to change all elements to elm. Otherwise we raise an error.
\begin{verbatim}
  | Replicate (n, a, _, pos) ->
       let sz  = evalExp(n, vtab, ftab)
       let elm = evalExp(a, vtab, ftab)
       match sz with
          | IntVal size ->
              if size >= 0
              then ArrayVal( List.map (fun x -> elm) [0..size-1], Int )
              else let msg = sprintf "Error: In replicate call, size is negative: %i" size
                   raise (MyError(msg, pos))
          | _ -> raise (MyError("replicate argument is not a number: "+ppVal 0 sz, pos))
\end{verbatim}
~\\

When typechecking, we check the size-expression, and if it is of type Int we return a tuple consisting ofArray (elm\_type), where elm\_type is the type of the elements in the array, and the typed syntax tree.
\begin{verbatim}
    | Replicate (n, a, _, pos) ->
        let (sz_type, sz_exp_dec) = checkExp ftab vtab n
        if sz_type = Int then
            let (elm_type, elm_exp_dec) = checkExp ftab vtab a
            (Array (elm_type), 
                    Replicate (sz_exp_dec, elm_exp_dec, elm_type, pos))
        else 
           raise (MyError ("replicate: Argument not an int", pos))
\end{verbatim}
~\\

The translation of replicate to MIPS was based on the translation of iota, so we will not show all the code here. The essence of the translated MIPS is that the expression that calculates the size of the array is evaluated. If it is too small an error is raised, and otherwise the array is allocated. Then the expression that evaluates the thing to be copied into the array is evaluated. Next is a loop that copies that thing into each index of the array. \\

We will now show and explain the changes from the \texttt{iota} code. \\
The first change is that the expression that results in the value to be copied needs to be translated. The result is saved in register val\_reg and the translation is in a\_code.
\begin{verbatim}
      let val_reg  = newName "val_reg"
      let a_code = compileExp a_exp vtable val_reg
\end{verbatim}
~\\

Booleans and characters only take up 1 byte instead of the usual 4, so we need to deal with these cases. We pattern match on the size of the elements in the array. If the size is 1 we only save 1 byte at a time.
\begin{verbatim}
     let loop_replicate =
          match getElemSize elem_type with
              | One  -> [ Mips.SB (val_reg, addr_reg, "0") ]
              | Four -> [ Mips.SW (val_reg, addr_reg, "0") ]
\end{verbatim}
~\\

Likewise, if the size is 1 we only add 1 to the address where we copy to, while we add 4 to it if the size is 4, i.e. a word. 
\begin{verbatim}
      let loop_footer = 
          match getElemSize elem_type with
              | One  ->
                  [ Mips.ADDI (addr_reg, addr_reg, "1")
                  ; Mips.ADDI (i_reg, i_reg, "1")
                  ; Mips.J loop_beg
                  ; Mips.LABEL loop_end
                  ]
 
              | Four ->
                  [ Mips.ADDI (addr_reg, addr_reg, "4")
                  ; Mips.ADDI (i_reg, i_reg, "1")
                  ; Mips.J loop_beg
                  ; Mips.LABEL loop_end
                  ]
\end{verbatim}
~\\

Lastly we concatenate all these pieces of MIPS code together like we described earlier.
\begin{verbatim}
      n_code
       @ checksize
       @ dynalloc (size_reg, place, elem_type)
       @ a_code
       @ init_regs
       @ loop_header
       @ loop_replicate
       @ loop_footer
\end{verbatim}


\subsection*{Filter}
\begin{verbatim}
  | Filter (farg, arrexp, _, pos) ->
        let arr = evalExp(arrexp, vtab, ftab)
        
        match arr with
          | ArrayVal (lst,tp1) ->
            let func = (evalFunArg (farg, vtab, ftab, pos, lst))
            match func with
              | BoolVal funcc ->
                let mlst = List.filter (fun x -> funcc) lst
                ArrayVal (mlst, tp1)
              | otherwise       -> raise (MyError ("First function argument return type is not bool: "+ppVal 0 arr, pos))
          | otherwise          -> 
            raise (MyError ("Second argument of filter is not an array: "+ppVal 0 arr, pos))
\end{verbatim}

First, the array (arr) is extracted from the value table. Then arr is matched to extract the primitive value. Then the lambda expression is found using the evalFunArg, which is matched in order to unpack it,because the filter function expects the function argument to return bool and not the user defined Value. This will at the same time check to see if the function argument is the correct return type. In case the function argument return type is not bool, or if the expression argument is not an array, then exceptions are thrown.

Unfortunately we never got to test filter, because a Parser error occoured. The same error occoured when the Interpreter implementation was not implemented, so it must be a fault with the Parser. Yet, the steps to taken in the Parser and Lexer for replicate were the same steps taken for filter. 

\subsection*{Scan}
N/A

